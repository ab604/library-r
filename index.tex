% Options for packages loaded elsewhere
\PassOptionsToPackage{unicode}{hyperref}
\PassOptionsToPackage{hyphens}{url}
\PassOptionsToPackage{dvipsnames,svgnames,x11names}{xcolor}
%
\documentclass[
  letterpaper,
  DIV=11,
  numbers=noendperiod]{scrreprt}

\usepackage{amsmath,amssymb}
\usepackage{lmodern}
\usepackage{iftex}
\ifPDFTeX
  \usepackage[T1]{fontenc}
  \usepackage[utf8]{inputenc}
  \usepackage{textcomp} % provide euro and other symbols
\else % if luatex or xetex
  \usepackage{unicode-math}
  \defaultfontfeatures{Scale=MatchLowercase}
  \defaultfontfeatures[\rmfamily]{Ligatures=TeX,Scale=1}
\fi
% Use upquote if available, for straight quotes in verbatim environments
\IfFileExists{upquote.sty}{\usepackage{upquote}}{}
\IfFileExists{microtype.sty}{% use microtype if available
  \usepackage[]{microtype}
  \UseMicrotypeSet[protrusion]{basicmath} % disable protrusion for tt fonts
}{}
\makeatletter
\@ifundefined{KOMAClassName}{% if non-KOMA class
  \IfFileExists{parskip.sty}{%
    \usepackage{parskip}
  }{% else
    \setlength{\parindent}{0pt}
    \setlength{\parskip}{6pt plus 2pt minus 1pt}}
}{% if KOMA class
  \KOMAoptions{parskip=half}}
\makeatother
\usepackage{xcolor}
\setlength{\emergencystretch}{3em} % prevent overfull lines
\setcounter{secnumdepth}{5}
% Make \paragraph and \subparagraph free-standing
\ifx\paragraph\undefined\else
  \let\oldparagraph\paragraph
  \renewcommand{\paragraph}[1]{\oldparagraph{#1}\mbox{}}
\fi
\ifx\subparagraph\undefined\else
  \let\oldsubparagraph\subparagraph
  \renewcommand{\subparagraph}[1]{\oldsubparagraph{#1}\mbox{}}
\fi

\usepackage{color}
\usepackage{fancyvrb}
\newcommand{\VerbBar}{|}
\newcommand{\VERB}{\Verb[commandchars=\\\{\}]}
\DefineVerbatimEnvironment{Highlighting}{Verbatim}{commandchars=\\\{\}}
% Add ',fontsize=\small' for more characters per line
\usepackage{framed}
\definecolor{shadecolor}{RGB}{241,243,245}
\newenvironment{Shaded}{\begin{snugshade}}{\end{snugshade}}
\newcommand{\AlertTok}[1]{\textcolor[rgb]{0.68,0.00,0.00}{#1}}
\newcommand{\AnnotationTok}[1]{\textcolor[rgb]{0.37,0.37,0.37}{#1}}
\newcommand{\AttributeTok}[1]{\textcolor[rgb]{0.40,0.45,0.13}{#1}}
\newcommand{\BaseNTok}[1]{\textcolor[rgb]{0.68,0.00,0.00}{#1}}
\newcommand{\BuiltInTok}[1]{\textcolor[rgb]{0.00,0.23,0.31}{#1}}
\newcommand{\CharTok}[1]{\textcolor[rgb]{0.13,0.47,0.30}{#1}}
\newcommand{\CommentTok}[1]{\textcolor[rgb]{0.37,0.37,0.37}{#1}}
\newcommand{\CommentVarTok}[1]{\textcolor[rgb]{0.37,0.37,0.37}{\textit{#1}}}
\newcommand{\ConstantTok}[1]{\textcolor[rgb]{0.56,0.35,0.01}{#1}}
\newcommand{\ControlFlowTok}[1]{\textcolor[rgb]{0.00,0.23,0.31}{#1}}
\newcommand{\DataTypeTok}[1]{\textcolor[rgb]{0.68,0.00,0.00}{#1}}
\newcommand{\DecValTok}[1]{\textcolor[rgb]{0.68,0.00,0.00}{#1}}
\newcommand{\DocumentationTok}[1]{\textcolor[rgb]{0.37,0.37,0.37}{\textit{#1}}}
\newcommand{\ErrorTok}[1]{\textcolor[rgb]{0.68,0.00,0.00}{#1}}
\newcommand{\ExtensionTok}[1]{\textcolor[rgb]{0.00,0.23,0.31}{#1}}
\newcommand{\FloatTok}[1]{\textcolor[rgb]{0.68,0.00,0.00}{#1}}
\newcommand{\FunctionTok}[1]{\textcolor[rgb]{0.28,0.35,0.67}{#1}}
\newcommand{\ImportTok}[1]{\textcolor[rgb]{0.00,0.46,0.62}{#1}}
\newcommand{\InformationTok}[1]{\textcolor[rgb]{0.37,0.37,0.37}{#1}}
\newcommand{\KeywordTok}[1]{\textcolor[rgb]{0.00,0.23,0.31}{#1}}
\newcommand{\NormalTok}[1]{\textcolor[rgb]{0.00,0.23,0.31}{#1}}
\newcommand{\OperatorTok}[1]{\textcolor[rgb]{0.37,0.37,0.37}{#1}}
\newcommand{\OtherTok}[1]{\textcolor[rgb]{0.00,0.23,0.31}{#1}}
\newcommand{\PreprocessorTok}[1]{\textcolor[rgb]{0.68,0.00,0.00}{#1}}
\newcommand{\RegionMarkerTok}[1]{\textcolor[rgb]{0.00,0.23,0.31}{#1}}
\newcommand{\SpecialCharTok}[1]{\textcolor[rgb]{0.37,0.37,0.37}{#1}}
\newcommand{\SpecialStringTok}[1]{\textcolor[rgb]{0.13,0.47,0.30}{#1}}
\newcommand{\StringTok}[1]{\textcolor[rgb]{0.13,0.47,0.30}{#1}}
\newcommand{\VariableTok}[1]{\textcolor[rgb]{0.07,0.07,0.07}{#1}}
\newcommand{\VerbatimStringTok}[1]{\textcolor[rgb]{0.13,0.47,0.30}{#1}}
\newcommand{\WarningTok}[1]{\textcolor[rgb]{0.37,0.37,0.37}{\textit{#1}}}

\providecommand{\tightlist}{%
  \setlength{\itemsep}{0pt}\setlength{\parskip}{0pt}}\usepackage{longtable,booktabs,array}
\usepackage{calc} % for calculating minipage widths
% Correct order of tables after \paragraph or \subparagraph
\usepackage{etoolbox}
\makeatletter
\patchcmd\longtable{\par}{\if@noskipsec\mbox{}\fi\par}{}{}
\makeatother
% Allow footnotes in longtable head/foot
\IfFileExists{footnotehyper.sty}{\usepackage{footnotehyper}}{\usepackage{footnote}}
\makesavenoteenv{longtable}
\usepackage{graphicx}
\makeatletter
\def\maxwidth{\ifdim\Gin@nat@width>\linewidth\linewidth\else\Gin@nat@width\fi}
\def\maxheight{\ifdim\Gin@nat@height>\textheight\textheight\else\Gin@nat@height\fi}
\makeatother
% Scale images if necessary, so that they will not overflow the page
% margins by default, and it is still possible to overwrite the defaults
% using explicit options in \includegraphics[width, height, ...]{}
\setkeys{Gin}{width=\maxwidth,height=\maxheight,keepaspectratio}
% Set default figure placement to htbp
\makeatletter
\def\fps@figure{htbp}
\makeatother
\newlength{\cslhangindent}
\setlength{\cslhangindent}{1.5em}
\newlength{\csllabelwidth}
\setlength{\csllabelwidth}{3em}
\newlength{\cslentryspacingunit} % times entry-spacing
\setlength{\cslentryspacingunit}{\parskip}
\newenvironment{CSLReferences}[2] % #1 hanging-ident, #2 entry spacing
 {% don't indent paragraphs
  \setlength{\parindent}{0pt}
  % turn on hanging indent if param 1 is 1
  \ifodd #1
  \let\oldpar\par
  \def\par{\hangindent=\cslhangindent\oldpar}
  \fi
  % set entry spacing
  \setlength{\parskip}{#2\cslentryspacingunit}
 }%
 {}
\usepackage{calc}
\newcommand{\CSLBlock}[1]{#1\hfill\break}
\newcommand{\CSLLeftMargin}[1]{\parbox[t]{\csllabelwidth}{#1}}
\newcommand{\CSLRightInline}[1]{\parbox[t]{\linewidth - \csllabelwidth}{#1}\break}
\newcommand{\CSLIndent}[1]{\hspace{\cslhangindent}#1}

\KOMAoption{captions}{tableheading}
\makeatletter
\makeatother
\makeatletter
\@ifpackageloaded{bookmark}{}{\usepackage{bookmark}}
\makeatother
\makeatletter
\@ifpackageloaded{caption}{}{\usepackage{caption}}
\AtBeginDocument{%
\ifdefined\contentsname
  \renewcommand*\contentsname{Table of contents}
\else
  \newcommand\contentsname{Table of contents}
\fi
\ifdefined\listfigurename
  \renewcommand*\listfigurename{List of Figures}
\else
  \newcommand\listfigurename{List of Figures}
\fi
\ifdefined\listtablename
  \renewcommand*\listtablename{List of Tables}
\else
  \newcommand\listtablename{List of Tables}
\fi
\ifdefined\figurename
  \renewcommand*\figurename{Figure}
\else
  \newcommand\figurename{Figure}
\fi
\ifdefined\tablename
  \renewcommand*\tablename{Table}
\else
  \newcommand\tablename{Table}
\fi
}
\@ifpackageloaded{float}{}{\usepackage{float}}
\floatstyle{ruled}
\@ifundefined{c@chapter}{\newfloat{codelisting}{h}{lop}}{\newfloat{codelisting}{h}{lop}[chapter]}
\floatname{codelisting}{Listing}
\newcommand*\listoflistings{\listof{codelisting}{List of Listings}}
\makeatother
\makeatletter
\@ifpackageloaded{caption}{}{\usepackage{caption}}
\@ifpackageloaded{subcaption}{}{\usepackage{subcaption}}
\makeatother
\makeatletter
\@ifpackageloaded{tcolorbox}{}{\usepackage[many]{tcolorbox}}
\makeatother
\makeatletter
\@ifundefined{shadecolor}{\definecolor{shadecolor}{rgb}{.97, .97, .97}}
\makeatother
\makeatletter
\makeatother
\ifLuaTeX
  \usepackage{selnolig}  % disable illegal ligatures
\fi
\IfFileExists{bookmark.sty}{\usepackage{bookmark}}{\usepackage{hyperref}}
\IfFileExists{xurl.sty}{\usepackage{xurl}}{} % add URL line breaks if available
\urlstyle{same} % disable monospaced font for URLs
\hypersetup{
  pdftitle={University of Southampton Library R training},
  pdfauthor={Alistair Bailey},
  colorlinks=true,
  linkcolor={blue},
  filecolor={Maroon},
  citecolor={Blue},
  urlcolor={Blue},
  pdfcreator={LaTeX via pandoc}}

\title{University of Southampton Library R training}
\author{Alistair Bailey}
\date{Last Updated on 2023-11-29}

\begin{document}
\maketitle
\ifdefined\Shaded\renewenvironment{Shaded}{\begin{tcolorbox}[borderline west={3pt}{0pt}{shadecolor}, frame hidden, interior hidden, enhanced, boxrule=0pt, sharp corners, breakable]}{\end{tcolorbox}}\fi

\renewcommand*\contentsname{Table of contents}
{
\hypersetup{linkcolor=}
\setcounter{tocdepth}{2}
\tableofcontents
}
\bookmarksetup{startatroot}

\hypertarget{preface}{%
\chapter*{Preface}\label{preface}}
\addcontentsline{toc}{chapter}{Preface}

\markboth{Preface}{Preface}

This is a book contains R code and examples for addressing tasks faced
by the Bibliometrics Team at the University of Southampton Library
Service.

If you want a full guide to data analysis in R check out
\href{https://r4ds.hadley.nz/}{R for Data Science}

Some of the materials here are re-used from previous workshops I ran for
biologists from 2019/20 called
\href{https://ab604.github.io/docs/coding-together-2019/}{Coding
Together}.

If you are new to R, then the first thing to know is that R is a
programming language and RStudio is program for working with R called an
integrated development environment (IDE). You can use R without RStudio,
but not the other way around.

\href{https://cran.r-project.org/}{Download R here} and
\href{https://posit.co/download/rstudio-desktop/}{Download RStudio
Desktop here}.

\bookmarksetup{startatroot}

\hypertarget{getting-started}{%
\chapter{Getting started}\label{getting-started}}

This has no content yet. See Knuth (1984) for additional discussion of
literate programming.

\hypertarget{data-transformations}{%
\section{Data transformations}\label{data-transformations}}

\bookmarksetup{startatroot}

\hypertarget{data-wrangling-i}{%
\chapter{Data wrangling I}\label{data-wrangling-i}}

This section is work in progress

\begin{Shaded}
\begin{Highlighting}[]
\FunctionTok{library}\NormalTok{(tidyverse)}
\end{Highlighting}
\end{Shaded}

\begin{verbatim}
Warning: package 'ggplot2' was built under R version 4.3.2
\end{verbatim}

\begin{verbatim}
Warning: package 'dplyr' was built under R version 4.3.2
\end{verbatim}

\begin{verbatim}
Warning: package 'stringr' was built under R version 4.3.2
\end{verbatim}

\begin{verbatim}
Warning: package 'lubridate' was built under R version 4.3.2
\end{verbatim}

\begin{Shaded}
\begin{Highlighting}[]
\FunctionTok{library}\NormalTok{(openxlsx)}
\end{Highlighting}
\end{Shaded}

\begin{verbatim}
Warning: package 'openxlsx' was built under R version 4.3.2
\end{verbatim}

\begin{Shaded}
\begin{Highlighting}[]
\FunctionTok{library}\NormalTok{(janitor)}
\end{Highlighting}
\end{Shaded}

\hypertarget{tidying-data}{%
\section{Tidying data}\label{tidying-data}}

In this section we're going to do some more complicated transformations.
Let's remind ourselves of the definition of tidy data:

\begin{enumerate}
\def\labelenumi{\arabic{enumi}.}
\tightlist
\item
  Each variable must have its own column.
\item
  Each observation must have its own row.
\item
  Each value must have its own cell.
\end{enumerate}

We'll start with a table of published articles where each row is a set
of observations about each article.

\emph{What makes it untidy?}

It's untidy because here we are interested in the University of
Southampton (UoS) affiliations associated with each article.
Unfortunatley, all the individual affiliations for all the institutions
affiliated with each paper are combined in a single column rather than
their own columns. Therefore the values for each affiliation aren't in
their own cells.

Hence for the purposes of the question we wish to pose, \emph{``Which
UoS department is affiliated to each article?''} the table is untidy in
terms of rules 1 and 3.

\hypertarget{our-workflow}{%
\subsection{Our workflow}\label{our-workflow}}

Let's consider the steps we need to take to arrive at a table which is
tidy and contains the values for each UoS affiliation for each published
article:

\begin{enumerate}
\def\labelenumi{\arabic{enumi}.}
\tightlist
\item
  Import the table, check it and repair if necessary.
\item
  Seperate the affiliations into a new column (variable) for each
  affiliation associated with each article.
\end{enumerate}

The articles don't all have the same number of affiliations, so we will
have lots of columns with only a few values in. In other words a wide
table (lots of columns) but sparsely populated. What we want is a table
that has only the UoS affiliation observations.

What is our usual trick for obtaining only the observations of interest?
We \texttt{filter} the rows. But hang on, we can't do that unless we
have the each affiliation in a row, rather spread across lots of
columns. So what do?

Transforming a table so that columns become rows, or rows become columns
is called \textbf{\emph{pivoting}}. Here, having separated the
affiliations and created a wide table with a variable for each
affiliation we want to pivot from wide-to-long.

When we pivot wide-to-long, the column names become a new set of
observations in a new variable, and the values from each column become
another set of observations in a new variable. Everything else is
duplicated and hence we end up with lots of rows.

In this way we create a table on which we can filter the rows for the
UoS affiliations.

For some articles there may be multiple UoS affiliations, so we may wish
to pivot long-to-wide to create a final tidy table with a single row for
each article, but columns for each UoS affiliation.

\hypertarget{importing}{%
\subsection{Importing}\label{importing}}

Let's start with a small data set called
\texttt{OA-compliance-subset-2023-11-29.xlsx} from the repository

\begin{Shaded}
\begin{Highlighting}[]
\CommentTok{\# Use the read.xlsx to download and load the excel file into object dat}
\NormalTok{dat }\OtherTok{\textless{}{-}} \FunctionTok{read.xlsx}\NormalTok{(}\StringTok{"https://github.com/ab604/library{-}r/raw/main/data/OA{-}compliance{-}subset{-}2023{-}11{-}29.xlsx"}\NormalTok{)}
\end{Highlighting}
\end{Shaded}

Use the \texttt{glimpse()} function from \texttt{dplyr} to take a peak
at this data:

\begin{enumerate}
\def\labelenumi{\arabic{enumi}.}
\tightlist
\item
  We can see that the table has 20 rows with 35 columns of variables.
\item
  Looking at the data types, we can see the import function parsed two
  types of data: \texttt{\textless{}chr\textgreater{}} character,
  meaning strings of text and \texttt{\textless{}dbl\textgreater{}}
  double, meaning numerical data. However, looking at the variable
  names, some are dates and haven't been identified automatically.
\item
  There are lots of \texttt{NA} s, which are missing values. This may or
  may not matter.
\end{enumerate}

\begin{Shaded}
\begin{Highlighting}[]
\FunctionTok{glimpse}\NormalTok{(dat)}
\end{Highlighting}
\end{Shaded}

\begin{verbatim}
Rows: 20
Columns: 35
$ Faculty                          <chr> "FELS", "FEPS", "FEPS", "FELS", "FSS"~
$ Report.Date                      <dbl> 45201, 45243, 45201, 45215, 45229, 45~
$ Ref.OA.Compliance                <chr> "No", "Yes", "Yes", "Yes", "Yes", "Ye~
$ `Reason.for.Non-Compliance`      <chr> "No Pure Record", NA, NA, NA, NA, NA,~
$ Time.remaining.to.make.compliant <dbl> 16, NA, NA, NA, NA, NA, NA, NA, NA, N~
$ Notes                            <dbl> NA, NA, NA, NA, NA, NA, NA, NA, NA, N~
$ Authors                          <chr> "Aeschbach D.; Cohen D.A.; Lockyer B.~
$ Title                            <chr> "Spontaneous attentional failures ref~
$ Source.Title                     <chr> "Sleep Health", "Frontiers in Artific~
$ Document.Type                    <chr> "Article", "Conference paper", "Artic~
$ Publisher                        <chr> "Elsevier Inc.", "IOS Press BV", "Ame~
$ DOI                              <chr> "10.1016/j.sleh.2023.07.013", "10.323~
$ Hyperlink.DOI                    <chr> "http://dx.doi.org/10.1016/j.sleh.202~
$ Affiliations                     <chr> "Department of Sleep and Human Factor~
$ SCH1                             <chr> "PSY", "CS", "PA", "BS", "SES", "PSY"~
$ SCH2                             <dbl> NA, NA, NA, NA, NA, NA, NA, NA, NA, N~
$ SCH3                             <dbl> NA, NA, NA, NA, NA, NA, NA, NA, NA, N~
$ Accepted.Date                    <chr> "45129", NA, "45156", "11/09/23", "45~
$ Epublication.Date                <chr> "45176", "45198", "45174", "45196", "~
$ Publisher.OA                     <chr> "No", "Yes", "No", "Yes", "Yes", "Yes~
$ Publisher.Licence                <chr> NA, "CC BY-NC", NA, "CC BY", "CC BY",~
$ PURE.ID                          <chr> "No Record", NA, "143740848", "143621~
$ Version.in.Pure                  <chr> NA, NA, "Accepted manuscript, Version~
$ Licence.in.Pure                  <chr> NA, NA, "UoS AM", "CC BY", NA, NA, "C~
$ Funder.1                         <chr> "N/A", "EPSRC", "N/A", "MRC", "N/A", ~
$ Funder.2                         <chr> NA, NA, NA, NA, NA, NA, NA, NA, NA, "~
$ Funder.3                         <dbl> NA, NA, NA, NA, NA, NA, NA, NA, NA, N~
$ Data.Statement                   <chr> NA, "No", "Yes", "Yes", NA, NA, NA, N~
$ Rights.Rentention.Statement      <chr> NA, NA, NA, "Yes", NA, NA, NA, NA, NA~
$ UKRI.Compliant                   <chr> NA, "No", NA, "Yes", NA, NA, NA, NA, ~
$ OA.Status                        <chr> "No", "Gold", "Green", "Gold", "Gold"~
$ Correspondence.address           <chr> "D. Aeschbach; Department of Sleep an~
$ UoS.Corresponding.Author         <chr> "No", "No", "Yes", "Yes", "Yes", "No"~
$ Controlled.Publisher             <chr> "Elsevier", NA, "ACS", NA, NA, "Sprin~
$ `Hide.Column.-.DOI.Formula`      <chr> "http://dx.doi.org/10.1016/j.sleh.202~
\end{verbatim}

\hypertarget{lets-fix-the-dates}{%
\subsection{Let's fix the dates}\label{lets-fix-the-dates}}

We'll make a new object called \texttt{dat\_repaired} and use the
\texttt{mutate} function from \texttt{dpylr} along with
\texttt{convert\_to\_date} in the \texttt{janitor} package to convert
the number values into dates.

Then we'll take a look with the glimpse function and we can see these
look like dates in the year-month-day format and are identified as
\texttt{\textless{}date\textgreater{}} data types.

\begin{Shaded}
\begin{Highlighting}[]
\NormalTok{dat\_repaired }\OtherTok{\textless{}{-}}\NormalTok{ dat }\SpecialCharTok{|\textgreater{}} \FunctionTok{mutate}\NormalTok{(}\AttributeTok{Report.Date =} \FunctionTok{convert\_to\_date}\NormalTok{(Report.Date),}
                              \AttributeTok{Accepted.Date =} \FunctionTok{convert\_to\_date}\NormalTok{(Accepted.Date),}
                              \AttributeTok{Epublication.Date =} \FunctionTok{convert\_to\_date}\NormalTok{(Epublication.Date))}
\FunctionTok{glimpse}\NormalTok{(dat\_repaired)}
\end{Highlighting}
\end{Shaded}

\begin{verbatim}
Rows: 20
Columns: 35
$ Faculty                          <chr> "FELS", "FEPS", "FEPS", "FELS", "FSS"~
$ Report.Date                      <date> 2023-10-02, 2023-11-13, 2023-10-02, ~
$ Ref.OA.Compliance                <chr> "No", "Yes", "Yes", "Yes", "Yes", "Ye~
$ `Reason.for.Non-Compliance`      <chr> "No Pure Record", NA, NA, NA, NA, NA,~
$ Time.remaining.to.make.compliant <dbl> 16, NA, NA, NA, NA, NA, NA, NA, NA, N~
$ Notes                            <dbl> NA, NA, NA, NA, NA, NA, NA, NA, NA, N~
$ Authors                          <chr> "Aeschbach D.; Cohen D.A.; Lockyer B.~
$ Title                            <chr> "Spontaneous attentional failures ref~
$ Source.Title                     <chr> "Sleep Health", "Frontiers in Artific~
$ Document.Type                    <chr> "Article", "Conference paper", "Artic~
$ Publisher                        <chr> "Elsevier Inc.", "IOS Press BV", "Ame~
$ DOI                              <chr> "10.1016/j.sleh.2023.07.013", "10.323~
$ Hyperlink.DOI                    <chr> "http://dx.doi.org/10.1016/j.sleh.202~
$ Affiliations                     <chr> "Department of Sleep and Human Factor~
$ SCH1                             <chr> "PSY", "CS", "PA", "BS", "SES", "PSY"~
$ SCH2                             <dbl> NA, NA, NA, NA, NA, NA, NA, NA, NA, N~
$ SCH3                             <dbl> NA, NA, NA, NA, NA, NA, NA, NA, NA, N~
$ Accepted.Date                    <date> 2023-07-22, NA, 2023-08-18, 2011-09-~
$ Epublication.Date                <date> 2023-09-07, 2023-09-29, 2023-09-05, ~
$ Publisher.OA                     <chr> "No", "Yes", "No", "Yes", "Yes", "Yes~
$ Publisher.Licence                <chr> NA, "CC BY-NC", NA, "CC BY", "CC BY",~
$ PURE.ID                          <chr> "No Record", NA, "143740848", "143621~
$ Version.in.Pure                  <chr> NA, NA, "Accepted manuscript, Version~
$ Licence.in.Pure                  <chr> NA, NA, "UoS AM", "CC BY", NA, NA, "C~
$ Funder.1                         <chr> "N/A", "EPSRC", "N/A", "MRC", "N/A", ~
$ Funder.2                         <chr> NA, NA, NA, NA, NA, NA, NA, NA, NA, "~
$ Funder.3                         <dbl> NA, NA, NA, NA, NA, NA, NA, NA, NA, N~
$ Data.Statement                   <chr> NA, "No", "Yes", "Yes", NA, NA, NA, N~
$ Rights.Rentention.Statement      <chr> NA, NA, NA, "Yes", NA, NA, NA, NA, NA~
$ UKRI.Compliant                   <chr> NA, "No", NA, "Yes", NA, NA, NA, NA, ~
$ OA.Status                        <chr> "No", "Gold", "Green", "Gold", "Gold"~
$ Correspondence.address           <chr> "D. Aeschbach; Department of Sleep an~
$ UoS.Corresponding.Author         <chr> "No", "No", "Yes", "Yes", "Yes", "No"~
$ Controlled.Publisher             <chr> "Elsevier", NA, "ACS", NA, NA, "Sprin~
$ `Hide.Column.-.DOI.Formula`      <chr> "http://dx.doi.org/10.1016/j.sleh.202~
\end{verbatim}

\bookmarksetup{startatroot}

\hypertarget{data-wrangling-ii}{%
\chapter{Data wrangling II}\label{data-wrangling-ii}}

This chapter has no content yet.

\bookmarksetup{startatroot}

\hypertarget{visualisation}{%
\chapter{Visualisation}\label{visualisation}}

This has no content yet.

\bookmarksetup{startatroot}

\hypertarget{references}{%
\chapter*{References}\label{references}}
\addcontentsline{toc}{chapter}{References}

\markboth{References}{References}

\hypertarget{refs}{}
\begin{CSLReferences}{1}{0}
\leavevmode\vadjust pre{\hypertarget{ref-knuth84}{}}%
Knuth, Donald E. 1984. {``Literate Programming.''} \emph{Comput. J.} 27
(2): 97--111. \url{https://doi.org/10.1093/comjnl/27.2.97}.

\end{CSLReferences}



\end{document}
